\label{Anexo}
\textbf{Procedimiento Desarrollo Seguro del Sistema}

Creo que esto va a integrarse en el desarrollo.

\textbf{Objetivo}

Asegurar el desarrollo seguro del producto de software, considerando el acceso físico y lógico no autorizado, las actualizaciones, la conexión a internet, el gateway, el sistema operativo, la dockerización, el tipo de dispositivos y el particionado de la memoria.

\textbf{Responsables:}

Desarrolladores

Administradores de Sistemas

Equipo de Seguridad Informática

\textbf{1. Accesos Lógicos No Autorizados}

\textbf{1.1. Gestión de Cuentas de Usuario:}

Establecer políticas para la creación y gestión de cuentas de usuario.

Aplicar el principio de privilegios mínimos.

\textbf{1.2. Monitoreo de Logs:}

Implementar un sistema de monitoreo de logs para detectar patrones de acceso no autorizado.

Analizar regularmente los logs para identificar posibles amenazas.

\textbf{2. Actualizaciones}

\textbf{2.1. Política de Actualizaciones:}

Establecer una política clara para la aplicación de actualizaciones de software y parches. 

Realizar actualizaciones periódicas y evaluar la necesidad de actualizaciones críticas.

\textbf{2.2. Pruebas Post-Actualización:}

Realizar pruebas exhaustivas después de cada actualización para asegurar la estabilidad del sistema.

\textbf{3. Conexión a Internet y Gateway}

\textbf{3.1. Firewalls y Filtros:}

Configurar firewalls para controlar el tráfico de red.

Utilizar filtros para bloquear tráfico malicioso.

\textbf{3.2. VPN Segura:}

Establecer conexiones a través de VPN seguras para proteger la comunicación entre sistemas.

\textbf{4. Sistema Operativo}

\textbf{4.1. Selección Segura del Sistema Operativo:}

Evaluar y seleccionar un sistema operativo que cumpla con los estándares de seguridad.

\textbf{5. Dockerización}

\textbf{5.1. Configuración Segura de Contenedores:}

Configurar los contenedores Docker con medidas de seguridad, como aislamiento de recursos.

\textbf{6. Tipo de Dispositivos}

\textbf{6.1. Evaluación de Dispositivos:}

Evaluar la seguridad de los dispositivos utilizados en el desarrollo.

Implementar políticas de uso seguro de dispositivos móviles.

\textbf{7. Particionado de la Memoria}

\textbf{7.1. Optimización y Particionado:}

Realizar un particionado adecuado de la memoria de los dispositivos para evitar vulnerabilidades.

Optimizar la asignación de recursos para mejorar el rendimiento y la seguridad.

\textbf{8. Revisiones y Actualizaciones del Procedimiento}

\textbf{8.1 Revisiones periódicas}

Revisar periódicamente el procedimiento para asegurar su relevancia y eficacia.

Actualizar el procedimiento según las cambiantes necesidades de seguridad.

\textbf{9. Acceso Físico No Autorizado}

\textbf{9.1. Control de Acceso Físico:}

Limitar el acceso físico a los servidores y equipos de desarrollo.

\textbf{9.2. Registro de Accesos:}

Mantener un registro de las personas autorizadas que ingresan a las áreas de desarrollo.

Revisar regularmente los registros para identificar cualquier acceso no autorizado.
