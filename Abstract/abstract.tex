% ************************** Thesis Abstract *****************************
% Use `abstract' as an option in the document class to print only the titlepage and the abstract.
\begin{abstract}

    En los últimos años, hemos observado un crecimiento constante de los dispositivos IoT y su integración en la vida de las personas e industrias. La heterogeneidad tecnológica y la cantidad de dispositivos actuales de IoT, complican la incorporación de medidas efectivas de seguridad para garantizar su administración, seguimiento, actualización, supervisión e intercomunicación. Las infraestructuras actuales plantean retos significativos para la implementación de medidas de seguridad, como lo son: la falta de recursos, la ausencia de sistemas de actualización seguros, la falta de procedimientos, la insuficiencia de barreras físicas y/o las contraseñas débiles. Cada uno de estos factores hacen a los dispositivos susceptibles a amenazas ya sea por sí sólos o combinados. 
    
    Entre las amenazas más crecientes se encuentra "supply chain attack" (ataque a la cadena de suministro). Supply chain attack, es capaz  de comprometer la seguridad de los dispositivos y sistemas en distintas etapas del ciclo de vida, despliegue y operación. Esto puede implicar el robo de datos, la introducción de malware o cualquier otra actividad maliciosa. Otra de las amenazas que podemos notar, es el fenómeno del "bricking" (bloqueo malicioso). Este bloqueo involucra la manipulación de dispositivos de manera que se vuelvan inoperables, a menudo mediante la eliminación del firmware o configuraciones críticas, que permiten a los atacantes forzar un apagado no autorizado o incontrolado de un dispositivo, potencialmente causando la pérdida de datos o el mal funcionamiento.
     
    Esta tesis se enfoca en la implementación de DevSecOps en IoT para mitigar múltiples amenazas a la seguridad. DevSecOps es una estrategia integral que fusiona desarrollo, seguridad y operaciones desde el inicio hasta el final del proyecto, incorporando la seguridad desde las primeras etapas. Al integrar de manera diligente las últimas actualizaciones de seguridad, se garantiza una postura defensiva proactiva contra las amenazas emergentes y se optimiza la resiliencia del sistema. La adopción apasionada de estos retos de seguridad, no solo refleja un compromiso con la excelencia técnica, sino que también, es un imperativo estratégico para salvaguardar la confiabilidad y la seguridad en los ambientes tecnológicos dinámicos de la actualidad.
    
    {	\bf Palabras clave: DevSecOps, IoT, MQTT	}
    \end{abstract}